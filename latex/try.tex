\documentclass[reqno]{amsart}
\usepackage{hyperref}
\usepackage{fullpage}
\usepackage{amsrefs}
\usepackage{tikz}
\usetikzlibrary{arrows}
\usepackage{newverbs}
\usepackage{fancyvrb}

\newif\ifshowkeys
\showkeysfalse

\ifshowkeys
\newcommand{\lbl}[1]{\label{#1}\textup{[\texttt{#1}]}\ \\}
\else
\newcommand{\lbl}{\label}
\fi

\input xypic
\newdir{ >}{{}*!/-9pt/\dir{>}}

\definecolor{maplecyan}{rgb}{0,1,1}
\definecolor{maplegreen}{rgb}{0.04,0.98,0.04}
\definecolor{maplemagenta}{rgb}{1,0.02,1}
\definecolor{mapleblue}{rgb}{0,0,1}
\definecolor{maplered}{rgb}{1,0,0}
\definecolor{maplepurple}{rgb}{0.5,0,0.5}

\definecolor{olivegreen}{cmyk}{0.64,0,0.95,0.40}
\definecolor{rawsienna}{cmyk}{0,0.72,1,0.45}

\newcommand{\Aut}	{\operatorname{Aut}}
\newcommand{\Hom}	{\operatorname{Hom}}
\newcommand{\Inn}	{\operatorname{Inn}}
\newcommand{\Out}	{\operatorname{Out}}

\newcommand{\alg}	{\operatorname{alg}}
\newcommand{\stab}	{\operatorname{stab}}

\newcommand{\Dl}        {\Delta}
\newcommand{\Gm}        {\Gamma}
\newcommand{\Lm}        {\Lambda}
\newcommand{\Sg}        {\Sigma}
\newcommand{\Om}        {\Omega}

\newcommand{\al}        {\alpha}
\newcommand{\bt}        {\beta} 
\newcommand{\gm}        {\gamma}
\newcommand{\dl}        {\delta}
\newcommand{\ep}        {\epsilon}
\newcommand{\zt}        {\zeta}
\newcommand{\ztb}       {\overline{\zt}}
\newcommand{\tht}       {\theta}
\newcommand{\lm}        {\lambda}
\newcommand{\om}        {\omega}
\newcommand{\sg}        {\sigma}

\newcommand{\xla}       {\xleftarrow}
\newcommand{\xra}       {\xrightarrow}

\newcommand{\Z}         {{\mathbb{Z}}}
\newcommand{\Q}         {{\mathbb{Q}}}
\newcommand{\R}         {{\mathbb{R}}}
\newcommand{\C}         {{\mathbb{C}}}

\newcommand{\ov}[1]     {\overline{#1}}
\newcommand{\ip}[1]     {\langle #1\rangle}
\newcommand{\st}        {\;|\;}
\newcommand{\tm}        {\times}
\newcommand{\sm}        {\setminus}
\newcommand{\bbm}       {\left[\begin{matrix}}
\newcommand{\ebm}       {\end{matrix}\right]}
\newcommand{\Wedge}     {\vee}
\newcommand{\sse}       {\subseteq}
\newcommand{\half}      {\tfrac{1}{2}}

\newcommand{\tA}	{\widetilde{A}}
\newcommand{\tC}	{\widetilde{C}}
\newcommand{\tF}	{\widetilde{F}}
\newcommand{\tG}	{\widetilde{G}}
\newcommand{\tU}	{\widetilde{U}}
\newcommand{\tX}	{\widetilde{X}}
\newcommand{\ta}	{\widetilde{a}}
\newcommand{\tc}	{\widetilde{c}}
\newcommand{\tq}	{\widetilde{q}}
\newcommand{\tu}	{\widetilde{u}}

\newcommand{\CX}        {\mathcal{X}}

\newcommand{\ab}	{\overline{a}}
\newcommand{\bb}	{\overline{b}}
\newcommand{\ub}        {\overline{u}}
\newcommand{\vb}	{\overline{v}}
\newcommand{\zb}	{\overline{z}}
\newcommand{\alb}	{\overline{\alpha}}
\newcommand{\lmb}	{\overline{\lambda}}

\newcommand{\hc}        {\widehat{c}}
\newcommand{\pp}        {\hphantom{+}}
\newcommand{\rt}        {\sqrt{2}}

\newcommand{\sss}{\scriptscriptstyle}
\renewcommand{\ss}{\scriptstyle}
\renewcommand{\:}{\colon}

\newcommand{\uc}        {\uncover}

\newtheorem{theorem}{Theorem}[section]
\newtheorem{conjecture}[theorem]{Conjecture}
\newtheorem{lemma}[theorem]{Lemma}
\newtheorem{proposition}[theorem]{Proposition}
\newtheorem{corollary}[theorem]{Corollary}
\theoremstyle{definition}
\newtheorem{remark}[theorem]{Remark}
\newtheorem{definition}[theorem]{Definition}
\newtheorem{example}[theorem]{Example}

\newtheorem{notation}{Notation}
\renewcommand{\thenotation}{} % make the notation environment unnumbered

\title{An interesting surface of genus two}
\author{Neil Strickland \\
(with computational help from Gemma Halliwell)
}

\renewcommand{\ttdefault}{pcr}

\DefineVerbatimEnvironment{checks}{Verbatim}{formatcom=\color{blue},fontfamily=courier}
\DefineVerbatimEnvironment{mcodeblock}{Verbatim}{formatcom=\color{red}}
\newverbcommand{\mcode}{\color{red}}{}
\newverbcommand{\fname}{\color{green}}{}

\begin{document}

 \begin{center}
  \begin{tikzpicture}[scale=3.1]
   \def\ya{0.7}
   \def\yb{1.4}
   \def\yc{2.1}
   \def\xa{0.8}
   \def\xb{1.6}
   \def\Da{( 0.0, 0.0)}
   \def\Ba{(-\xa, \ya)}
   \def\Bb{( \xa, \ya)}
   \def\Ca{(   0, \ya)}
   \def\Aa{(-\xb, \yb)}
   \def\Ab{( \xa, \yb)}
   \def\Ac{(-\xa, \yb)}
   \def\Ad{( \xb, \yb)}
   \def\Za{(   0, \yb)}
   \def\Ta{(   0, \yc)}
   \begin{scope}[xshift=1cm]
    \draw \Da node{$H_{10}=D_8$};
    \draw \Ca node{$H_3=\ip{\lm}$};
    \draw \Ba node{$H_8=\ip{\lm^2,\mu}$};
    \draw \Bb node{$H_9=\ip{\lm^2,\lm\mu}$};
    \draw \Aa node{$H_6=\ip{\lm^2\mu}$};
    \draw \Ab node{$H_5=\ip{\lm\mu}$};
    \draw \Ac node{$H_4=\ip{\mu}$};
    \draw \Ad node{$H_7=\ip{\lm^3\mu}$};
    \draw \Za node{$H_2=\ip{\lm^2}$};
    \draw \Ta node{$H_1=1$};
    \draw[<-,shorten <=15pt,shorten >=15pt] \Da -- \Ba;
    \draw[<-,shorten <=15pt,shorten >=15pt] \Da -- \Bb;
    \draw[<-,shorten <=15pt,shorten >=15pt] \Da -- \Ca;
    \draw[<-,shorten <=15pt,shorten >=15pt] \Ca -- \Za;
    \draw[<-,shorten <=15pt,shorten >=15pt] \Ba -- \Aa;
    \draw[<-,shorten <=15pt,shorten >=15pt] \Ba -- \Ac;
    \draw[<-,shorten <=15pt,shorten >=15pt] \Ba -- \Za;
    \draw[<-,shorten <=15pt,shorten >=15pt] \Bb -- \Ab;
    \draw[<-,shorten <=15pt,shorten >=15pt] \Bb -- \Ad;
    \draw[<-,shorten <=15pt,shorten >=15pt] \Bb -- \Za;
    \draw[<-,shorten <=19pt,shorten >=15pt] \Aa -- \Ta;
    \draw[<-,shorten <=15pt,shorten >=15pt] \Ab -- \Ta;
    \draw[<-,shorten <=15pt,shorten >=15pt] \Ac -- \Ta;
    \draw[<-,shorten <=19pt,shorten >=15pt] \Ad -- \Ta;
    \draw[<-,shorten <=15pt,shorten >=15pt] \Za -- \Ta;
   \end{scope}
  \end{tikzpicture}
 \end{center}

\end{document}

 \begin{center}
  \begin{tikzpicture}[scale=2.7]
   \begin{scope}[xshift=4cm]
    \draw \Da node{$\Q$};
    \draw \Ca node{$\Q(\sqrt{14})$};
    \draw \Ba node{$\Q(\sqrt{2})$};
    \draw \Bb node{$\Q(\sqrt{7})$};
    \draw \Aa node{$\Q(\al-\bt)$};
    \draw \Ab node{$\Q(\bt)$};
    \draw \Ac node{$\Q(\al+\bt)$};
    \draw \Ad node{$\Q(\al)$};
    \draw \Za node{$\Q(\sqrt{2},\sqrt{7})$};
    \draw \Ta node{$K$};
    \draw[->,shorten <=11pt,shorten >=11pt] \Da -- \Ba;
    \draw[->,shorten <=11pt,shorten >=11pt] \Da -- \Bb;
    \draw[->,shorten <=11pt,shorten >=11pt] \Da -- \Ca;
    \draw[->,shorten <=11pt,shorten >=11pt] \Ca -- \Za;
    \draw[->,shorten <=11pt,shorten >=11pt] \Ba -- \Aa;
    \draw[->,shorten <=11pt,shorten >=11pt] \Ba -- \Ac;
    \draw[->,shorten <=11pt,shorten >=11pt] \Ba -- \Za;
    \draw[->,shorten <=11pt,shorten >=11pt] \Bb -- \Ab;
    \draw[->,shorten <=11pt,shorten >=11pt] \Bb -- \Ad;
    \draw[->,shorten <=11pt,shorten >=11pt] \Bb -- \Za;
    \draw[->,shorten <=11pt,shorten >=11pt] \Aa -- \Ta;
    \draw[->,shorten <=11pt,shorten >=11pt] \Ab -- \Ta;
    \draw[->,shorten <=11pt,shorten >=11pt] \Ac -- \Ta;
    \draw[->,shorten <=11pt,shorten >=11pt] \Ad -- \Ta;
    \draw[->,shorten <=11pt,shorten >=11pt] \Za -- \Ta;
   \end{scope}
  \end{tikzpicture}
 \end{center}


\end{document}


